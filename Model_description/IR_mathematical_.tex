% Options for packages loaded elsewhere
\PassOptionsToPackage{unicode}{hyperref}
\PassOptionsToPackage{hyphens}{url}
%
\documentclass[
]{article}
\usepackage{amsmath,amssymb}
\usepackage{iftex}
\ifPDFTeX
  \usepackage[T1]{fontenc}
  \usepackage[utf8]{inputenc}
  \usepackage{textcomp} % provide euro and other symbols
\else % if luatex or xetex
  \usepackage{unicode-math} % this also loads fontspec
  \defaultfontfeatures{Scale=MatchLowercase}
  \defaultfontfeatures[\rmfamily]{Ligatures=TeX,Scale=1}
\fi
\usepackage{lmodern}
\ifPDFTeX\else
  % xetex/luatex font selection
\fi
% Use upquote if available, for straight quotes in verbatim environments
\IfFileExists{upquote.sty}{\usepackage{upquote}}{}
\IfFileExists{microtype.sty}{% use microtype if available
  \usepackage[]{microtype}
  \UseMicrotypeSet[protrusion]{basicmath} % disable protrusion for tt fonts
}{}
\makeatletter
\@ifundefined{KOMAClassName}{% if non-KOMA class
  \IfFileExists{parskip.sty}{%
    \usepackage{parskip}
  }{% else
    \setlength{\parindent}{0pt}
    \setlength{\parskip}{6pt plus 2pt minus 1pt}}
}{% if KOMA class
  \KOMAoptions{parskip=half}}
\makeatother
\usepackage{xcolor}
\usepackage[margin=1in]{geometry}
\usepackage{graphicx}
\makeatletter
\def\maxwidth{\ifdim\Gin@nat@width>\linewidth\linewidth\else\Gin@nat@width\fi}
\def\maxheight{\ifdim\Gin@nat@height>\textheight\textheight\else\Gin@nat@height\fi}
\makeatother
% Scale images if necessary, so that they will not overflow the page
% margins by default, and it is still possible to overwrite the defaults
% using explicit options in \includegraphics[width, height, ...]{}
\setkeys{Gin}{width=\maxwidth,height=\maxheight,keepaspectratio}
% Set default figure placement to htbp
\makeatletter
\def\fps@figure{htbp}
\makeatother
\setlength{\emergencystretch}{3em} % prevent overfull lines
\providecommand{\tightlist}{%
  \setlength{\itemsep}{0pt}\setlength{\parskip}{0pt}}
\setcounter{secnumdepth}{-\maxdimen} % remove section numbering
\ifLuaTeX
  \usepackage{selnolig}  % disable illegal ligatures
\fi
\usepackage{bookmark}
\IfFileExists{xurl.sty}{\usepackage{xurl}}{} % add URL line breaks if available
\urlstyle{same}
\hypersetup{
  hidelinks,
  pdfcreator={LaTeX via pandoc}}

\author{}
\date{\vspace{-2.5em}}

\begin{document}

MODELING THE ASSOCITATION GENOTYPE AND PHENOTYPE in ANOPHELES MOSQUITOES
IN AFRICA

\subsection{I- Data inputs}\label{i--data-inputs}

• WHO discriminating concentration bioassay results from the Vector
Atlas database,incorporating data from IR Mapper and WHO Malaria Threats
Map. • Vector Atlas data results about genotype frequency • Covariate
data : - Modelled use of ITNs 2000-2022 (MAP, see Bertozzi-villa et
al.~2021) - Reported coverage of IRS 1997-2022 (WHO/MAP) - Population
density 2000-2030 (Worldpop/MAP, adjusted to UN forecasts 2021-2030)

\subsection{II - Observation models}\label{ii---observation-models}

\subsection{1- Bioassay data}\label{bioassay-data}

Given data on \(N_{i,j}\) mosquitoes tested and \(D_(i,j)\) deaths
observed, we assume these counts follow a Beta--Binomial distribution:
and , the observed numbers of mosquitoes tested, and that died,we assume
these follow a beta-binomial distribution:

\[
D_{j,i} \sim Betabinomial(N, Q*, \rho)
\]

Where \(Q*\) is the modelled population-level susceptibility of the
total vector population at the location \(l\) and time \(t\) from which
this sample of mosquitoes was collected. The overdispersion parameter
\(\rho\) is specific to the insecticide class \(c(j)\) of insectcide
\(j\), capturing the non-independence of individuals mosquitoes within
the same bioassay sample. This accounts for the extra variability in
mortality observed across different samples, reflecting the inherent
heterogeneity in susceptibility measurements.

\subsubsection{Genotype test results for
multilocus:}\label{genotype-test-results-for-multilocus}

Let \(M_z\)denote the number of mosquitoes tested and \(N_z\) the number
positive for genotype \(g\) in a given record. Here, \(Z\) represents
the vector of genotype frequencies in the population at the location and
time of sampling. When multiple genotype categories are considered, we
model these counts using a Dirichlet--Multinomial distribution, which
accounts for both the multinomial sampling of genotypes and extra
variability (overdispersion) in the population frequencies.

\[
N^{z} \sim DirMultinomial( M^{z}, Z, \rho_z)
\]

\subsubsection{Genotype test results for single
locus:}\label{genotype-test-results-for-single-locus}

Let \(N_{lo}^u\) denote the number of mosquitoes positive for a
particular genotype at locus \(lo\), and let \(M^{u}\) be the total
number of mosquitoes tested. Assuming Hardy--Weinberg equilibrium (HWE)
with resistant-allele frequency \(p\) (so \[𝑞= 1 − p)\] and
\[q = 1 − p\] for the susceptible allele), the expected genotype
frequencies in the population are given by

\[(U_{l_0,t} = p_{lo,t}^2, 2p_{lo,t} q_{lo,t}, q_{lo,t}^2)\]

And the over-dispersion \(rho_{u}\) via a Dirichlet--Multinomial: \[
N^{u} \sim DirMultinomial (M^{u}, U, \rho_{u})
\]

\subsubsection{Allele frequency data:}\label{allele-frequency-data}

We consider \(N^{i}\) is the number of mosquitoes tested at a gien time
point \(t\), and let \(P_{lo}\) be the allele frequency of a specific
allele \(a\) at locus \(lo\). We define \(N_{lo,t}^{a}\) as the number
of carrying allele \(a\) in the sample. These countes are modeled using
a Betabinomial likelihood to account for sampling variability and the
overdispersion :

\[  
N_{lo,t}^{a} \sim Betabinomial (N^{a}, P_{lo}, \rho_{a})
\]

\subsection{II - Process models}\label{ii---process-models}

\subsubsection{1- Computing phenotype
frequencies:}\label{computing-phenotype-frequencies}

The model computes the mortality probability \(Q^{*}_{lo,t, c}\) for
each insecticide \(c\) at time \(t\) as a weighted mixture of
genotype-specific utilities. For each genotype \(g\) at locus \(lo\),
the locus-specific susceptibility contribution \(u_{g,lo,c}\) is
calculated based on allelic configuration and dominance \(ℎ_{lo}\) :

\[
u_{g,lo,c} = 1 * R_{g,lo} + \theta_{lo,c} * (1 - L_{g,lo}) + (\theta_{lo,c} * ℎ_{lo} + (1 - h_{lo}) (L_{g,lo} - R_{g,lo}))
\]

where: \(L_{g,lo}\) and \(R_{g,lo}\) : are indicators for the allelic
states of genotype \(g\) at locus \(lo\) (SS and RR, respectively),
\(\theta_{lo,c} \in [0,1]\) is the locus-specific resistance effect (how
much the locus reduces susceptibility to insecticide \(c\),
\(ℎ_{lo} \in [0,1]\) is the dominance coefficient, representing the
fraction of the resistant allele effect expressed in heterozygotes. The
overall susceptibility of genotype \(g\) to insecticide \(c\) is
obtained by multiplying across loci and scaling by the baseline
mortality of the wild-type genotype \(v_{c}^{s} \in [0,1]\):

\[
U_{g,c} = v_{c}^{s} * \prod_{lo=1}^{n} u_{g,lo,c}
\]

At each time step \(t\), the population-level fraction of susceptible
mosquitoes is then computed as the weighted average over genotype
frequencies \(Z_{g,lo,t}\) (which sum to 1):

\[
Q_{l,t,c}^{*} = \sum_{g=1}^{n^{3}} U_{g,c} Z_{g,lo,t}
\]

This formulation captures how allelic states at multiple loci,
locus-specific resistance effects, and dominance combine to determine
both genotype-specific mortality and the population-level susceptibility
to each insecticide.

\subsubsection{2- Computing the probability of having a specific
genotype}\label{computing-the-probability-of-having-a-specific-genotype}

We construct the multilocus genotype frequency \(Z_{g,lo,t}\) as the
product of per-locus, genotype probabilities\(F_{g,lo,t}\) across the
nloci: \[Z_{g,t} = \prod_{l=1}^{n} F_{g,lo,t}
\] Let \(P_{lo,t}\) denote the frequency of allele resistant at locus lo
at time t (so \(1-P_{lo,t}\) is the frequency of allele S). For each
multilocus genotype g and locus lo, define left and right allele
indicator dummy matrices L and R (S =1, R=0). The per-locus probability
that matches the genotype of g is
\[F_{g,lo,t} = (1 - P_{lo, t})L_{g, lo} + P_{lo, t} (1 - L_{g, lo}) *
(1 - P_{lo, t})R_{g, lo} + P_{lo, t} (1 - R_{g, lo}) * (1 + L_{g, lo} - R_{g, lo})
\] \#\#\# Computing the frequency of the genotype at the next time
series The genotype will be updated by normalization. At each time step
we form an unnormalized genotype mass by multiplying the genotype prior
(from allele frequencies) with its multilocus fitness. Concretely, with
\[G_{g,lo,t} = 3^{B}\] genotypes over \(B\) loci, so
\[G_{g,lo,t} = 1 * R_{g, lo} + w_{lo} * (1 - L_{g, lo}) + 
        (w_{lo} * h_{lo} + 1 - h_{lo}) (L_{g, lo} - R_{g, lo})
        \] The prior for genotype gat time tis \(Z_{g,t}\) (product of
locus-wise left/right allele probabilities, accounting for the \(SR\)
duplicate via \(1+L-R\)); the genotype fitness is
\[r_{g,t} = \prod_{lo}^{n} G_{g,lo,t}\] ) The unnormalized next-step
mass is \[Z_{g,t+1}^{*} = Z_{g,t} * r_{g,t}\] We then normalize across
genotypes, \[Z_{g,t+1} = Z_{g,t+1}^{*}/ \sum_{g=1}^{n^{3}} Z_{g,t+1}^{*}
\] to obtain a valid probability distribution that sums to 1.

\subsubsection{Update the multilocus allele
frequency}\label{update-the-multilocus-allele-frequency}

For each locus \(lo\), we compute the next-step resistant-allele
frequency as the posterior-weighted average number of resistant alleles
per diploid, divided by 2. Concretely, \(Z_{g,t+1}\) is the normalized
posterior probability of genotype \(g\) at time \(t+1\) (so
\(\sum_{g} Z_{g, t+1}=1\)), and \(L_{g,lo}\),\(R_{g,lo}∈{0,1}\) encode
the left and right alleles at that locus with \(1 = susceptible (S)\)
and \(0 = resistant (R)\). Thus, \(1 - L_{g,lo}\) is an indicator that
the left allele is resistant, and \((1 - R_{g,lo})\) indicates the right
allele is resistant. The quantity \((1 - L_{g,lo})+(1 - R_{g,lo})\).
Averaging this count across genotypes with weights \(Z_{g,t+1}\) gives
the expected number of resistant alleles per individual at that locus;
dividing by 2 converts this expected count (out of two alleles) into a
frequency:
\[P_{lo, t+1} = 1/2  \sum_{g=1}^{n^{3}} Z_{g, t+1} (2 - L_{g, lo} - R_{g, lo})\]

\subsubsection{Selection pressure and relative
fitness}\label{selection-pressure-and-relative-fitness}

We model locus- and insecticide-specific selection pressures as
positively-constrained selection coefficients \(s_{j,lo,t}\),
representing selection towards resistance only; if no resistance driver
is present, the selection is zero. These coefficients are expressed as a
linear combination of observed covariates \(X_{k,lo,t}\), which quantify
the intensity of potential resistance drivers at each locus and time
(scaled from 0 to 1, e.g., from 0 for 0\% LLIN coverage to 1 for 100\%
coverage), and the positive selection effects \(γ_{j,k}\) for
insecticide \(j\) associated with driver \(k\) :

\[
s_{j,lo,t} = \sum_{k=1}^{K} X_{k,lo,t} * γ_{j,k}
\]

and the relative fitness,

\[
w_{j,lo,t} = 1 + s_{j,lo,t}
\]

The selection effects \(γ_{j,k}\) are constrained to be positive via
\[ γ_{j,k} =e xp(β_{j,k})\], and the coefficients \(β_{j,k}\) are
modeled hierarchically: first sharing information between insecticides
within the same class \(C(j)\), and then across classes,

\[
β_{j,k} ∼ N(β_{C(j),k}), \sigma_{k2}^{2}
\]

\[
β_{j,k} ∼ N(β_{C(j),k}), \sigma_{k1}^{2}
\]

which allows the model to learn about common drivers of resistance for
insecticides with sparse assay data and captures pleiotropic effects of
resistance-conferring genotypes across multiple insecticides.

\end{document}
